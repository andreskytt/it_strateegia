\chapter{Juhtimine}
\TODO \cite{svcthesis} kaudu IT Governance definitsioon. Ka slaididesse! (lk 26)
\section{Partnerite juhtimine}
Enne, kui asuda end lahutamatult sisse söönud partnerist, "puugist", lahutama, on kasulik veidi mõelda. Asi võib olla selles, et koostöö võib parasiitluse asemel siiski sümbioosiks osutuda. 

On hea teha järgnevat:
\begin{itemize}
	\item Vii kas ise läbi või telli audit hindamaks süsteemi kvaliteeti ning strateegilist positsiooni. Hinnata tuleks nii koodi kui sellist, arhitektuurset lahendust kui ka süsteemi üldist avatust (dokumentatsiooni tase, APIde olemasolu jne.). Audit on kasulik ka sõltumatu argumendina põhjendamaks potentsiaalselt kõrgeid kulusid, mis süsteemide ümber kirjutamise või asendamisega paratamatult kaasnevad (vt. ka \ref{sec:rooste})
	\item Hinda alternatiive. ``Puugist`` lahti saamine ei ole väga mõistlik, kui järgmisel hankel jälle vaid seesama ettevõte tõsiseltvõetava pakkumise suudab esitada  
\end{itemize}

\section{Kvaliteedi juhtimine}
Kuna süsteemide puhul on reeglina tegemist sotsiotehniliste süsteemidega (st. süsteemi osaks loetakse ka lõppkasutaja) on oluliseks süsteemi kvaliteedi parameetriks nende kahe osise omavahaeline koostöö. Samuti väljendub süsteemi kasulikkus reeglina läbi lõppkasutaja. Järelikult on kasutajamugavus ja -kogemus kvaliteedijuhtimise olulised objektid. Reeglina toimub kasutajakogmuse disain süsteemi loomise algfaasis ning kvaliteedi tagamine selle lõpus, mistõttu kaks kompetentsi, paraku, reeglina kokku ei puutu.  

Tulemi kvaliteet sõltub olulisel määral kasutatavast protsessist, nii ka tarkvaras. Vt. \ref{humble}.

\TODO Vigade skoorimine ja arendussaba sugemine. Vigade mudel ja testimise olulisus. Keskpärasusega harjumine


\section{Teenustaseme juhtimine}
Tavaliselt on organisatsioonides diskussiooni objektiks teenuse mõiste. Kuna kirjandusest samuti sobilikku määratlust võtta ei ole, siis võivad vaidlused osutuda pikkadeks ja viljatuteks. Seetõttu on oluline välja tuua teenuse definitsiooni kaks peamist struktuurset omadust lisaks slaididel kirjeldatud funktsionaalsetele omadustele:
\begin{itemize}
	\item Definitsioon peab suutma tõmmata selge, organisatsiooni ülevalt alla läbiva, piiri eri teenuste vahele
	\item Definitsioon peab olema vastuvõetav tervele organisatsioonile
\end{itemize}

samas:
\begin{itemize}
	\item Definitsioon ei pea olema täielik, määratletavad teenused ei pea katma kõiki organisatsiooni tegevusvaldkondi
	\item Definitsioon võib olla meelevaldne ("Teenuseks on rakendusse sisse logimine ja meie töötajate füüsiline kohalolu leti taga, miski muu teenus ei ole")
\end{itemize}

Kui esimesed kaks tingimust on täidetavad, võib vaidlused lõppenuks lugeda ning asuda teenustaset jutima.

\section{Projektijuhtimine}
Projektijuhtimine on väga keeruline teema ja sellest on kirjutatud palju. Siiski on mõned IT juhtimise vaatenurgast olulised asjad, millest on oluline aru saada. 

Võtkem näiteks projekti lõpptähtaja hindamise. Hindamine toimub tavaliselt nii, et projekt jagatakse sammudeks, iga sammu eest vastutaja annab oma hinnangu ja tulemus liidetakse kokku. Kõlab mõistlikult, eks? Kuid vaatame lähemalt. Iga vastutaja on tõenäoliselt kogenud inimene. Seega oskab ta anda hinnangu oma sammu mediaankestvust: kestvust, millest pooltel juhtudel õnnestub samm kiiremini läbida ja pooltel aeglasemalt. Eriti kogenud inimesed lisavad ka puhvri andes hinnangu, mis peab vett umbes neljal juhul viiest. Ühe sammu puhul on kõik hästi. Kuid kui samme on kaks? Et projekt ei hilineks, peaks oodatust kiiremini valmis saama \emph{mõlemad} sammud. Ehk, $P_{edu}=P(S_1\cap S_2)=P(S_1)P(S_2)=0.8\times0.8=0.64$. Üldisemalt, projekti hinnatust kiiremaks valmimiseks peab oodatust kiiremini valmima üks rohkem kui pooled sammud. Natuke arvutades saame, et viie sammu puhul on tõenäosus pisut üle poole ning kümne puhul juba veerandi ligi. Kokkuvõte on toodud tabelis \ref{tab:success}. Kuna vähegi keerukam it-projekt sisaldab sadu tegevusi\sidenote{Muidugi tuleb siin silmas pidada kriitilist ahelat, ehk ajaliselt pikimat üksteisest sõltuvate tegevuste jada projektis. Kuni muude tegevuste venimine uut kriitilist ahelat ei tekita, projekti lõpptähtaeg neist ei sõltu.} on üks olulisi projektide venimise põhjusi käes.

Juba paarikümne sammu puhul ütleb statistika, et projekti õigeaegselt või paremini lõppemise tõenäosus langeb kaheksa protsendi juurde. Ja seda eeldades, et sammude kestvuse jaotus on sümmeetriline. Mis aga kindlasti ei pea paika\footnote{\url{http://priceonomics.com/why-are-projects-always-behind-schedule/} toob ära jaotuse, mis põhineb 70 000 projektitundi sisaldava andmehulga analüüsil. Sealt on pärit ka toodud argumentatsiooni idee.}. 

Milles siis asi? Probleem on selles, et tõenäosuslikke muutujaid ei saa niisama lihtsasti liita. Tuleb aru saada, et hinnang projekti kestvusele on samuti tõenäosusjaotus nagu on seda ka üksikute sammude hinnangud. Kuna projekti tegevuste seosed on mittetriviaalsed, läheb jaotuste liitmine kiiresti liig keeruliseks. Agiilsed arendusmetoodikad keelduvadki tolle keerukusega tegelemast ja annavad lubadusi vaid lühikese ajaperioodi kohta. Siiski on ka mitte-agiilsete projektide puhul võimalik rääkida näiteks analüüsi, arenduse ja testimise sammudest millede jaotustega on võimalik tegeleda.  

Kindlasti aga tuleb vältida lihtsat hinnangute liitmist.

\begin{table}
	\begin{center}
\begin{tabular}{lll}

\toprule
Samme & Mediaan & 80\% hinnang \\
\midrule
1 & 0.5 & 0.8 \\
2 & 0.25 & 0.64 \\
3 & 0.25 & 0.64 \\
4 & 0.125 & 0.512 \\
5 & 0.125 & 0.512 \\
6 & 0.0625 & 0.4096 \\
7 & 0.0625 & 0.4096 \\
8 & 0.03125 & 0.32768 \\
9 & 0.03125 & 0.32768 \\
10 & 0.015625 & 0.262144 \\
11 & 0.015625 & 0.262144 \\
12 & 0.0078125 & 0.2097152 \\
13 & 0.0078125 & 0.2097152 \\
14 & 0.00390625 & 0.16777216 \\
15 & 0.00390625 & 0.16777216 \\
16 & 0.001953125 & 0.134217728 \\
17 & 0.001953125 & 0.134217728 \\
18 & 0.000976563 & 0.107374182 \\
19 & 0.000976563 & 0.107374182 \\
20 & 0.000488281 & 0.085899346 \\
21 & 0.000488281 & 0.085899346 \\
22 & 0.000244141 & 0.068719477 \\
23 & 0.000244141 & 0.068719477 \\
24 & 0.00012207 & 0.054975581 \\
25 & 0.00012207 & 0.054975581 \\

\bottomrule
\end{tabular}
		\caption{Projekti edukuse tõenäosus}
		\label{tab:success}

	\end{center}
\end{table}

\section{Reaalsusest, mudelitest ja juhtimisest}
Selleks, et midagi juhtida, peab meil olema mingit liiki mudel juhitavast objektist. Teeme ju eelduse, et muutes mõnda sisendit muutub väljund sobivas suunas. On oluline aru saada, et see mudel eksisteerib vaid meie peas ning on vaid reaalsuse lähendus. \citeauthor{box1976science} ütleb, et kõik mudelid on valed ja mõned mudelid on kasulikud\cite{box1976science}. Tema vana ütlemine on kahtlemata relevantne ka praegu ning artiklis toodud mudelid (kahtlemata valed kuid vahel osutunud kasulikuks) võimaldavad läheneda süstemaatiliselt objektide uurimisele. 

Lühidalt ütleb Box, et me kõigeapelt koostame teooria või mudeli reaalsuse kohta, siis disainime ja viime läbi eksperimendi mudeli valideerimiseks, saame uusi andmeid ning nende abil kohendame mudelit, millest tuleneb uue eksperimendi disain. Artiklis räägitakse küll teadlastest ja statistikutest kuid üldine lähenemine on  väga relevantne üldises keeruliste süsteemide juhtimise kontekstis: kirjeldatakse nii liigset teooriakesksust (vrdl. ``arhitektuursed astronaudid``) kui ka teooriapuudust ning rõhutatakse vajadust püsida kontaktis uuritava objektiga. Olulisimana jääb siiski kõlama vajadus eristada olemuslikult piiratud mudelit tema kirjeldatavast reaalsusest.  

\section{Küsimusi aruteluks}
\subsection{Kuidas vabaneda end sisse söönud ”puugist”?}
\TODO täida sisuga. Viide Aeda legacy tööle eespool. 
Saalist
\begin{itemize}
	\item Miks peaks? Ehk, mis on see konkreetne põhjus, miks vabaneda. Peeglisse vaatamise märkus. Ilma selle artikuleerimiseta on raske kulu õigustada. Ja peab olema selge, mis tulevikus teisiti läheb ning kuidas puuk tekkis
	\item Nullist kirjutada on võimalik vaid siis, kui on selged piirid tükkide vahel
	\item Peale nullist ümber kirjutamise ei ole ühtegi head ideed
\end{itemize}

\begin{itemize}
	\item Kirjuta kogu asi korstnasse. See ei pruugi olla palju kallim (arvutused ja modelleerimine!)
	\item Osta ära
	\item Investeeri teadlikult muskli kasvatamisse ja arvesta kvaliteedi langusega
\end{itemize}

\subsection{Mida saab teha saamatu äripoolega?}
Eespool juba arutelu "kas saab ITd mõistlikult juhtida, kui klient ei ole mõistlikult juhitud?". Erista teadmiste/oskuste puudus võimete puudmisest (organisatsiooni, mitte isiku tasemel) Lolli tuleb kas kiita või temast lahti saada. Do not try to teach a pig to sing. It wastes your time and annoys the pig. 

\TODO
Saalist
\begin{itemize}
	\item Koolita klienti. Koolitus (targutamine) kui agressioon. Tule, ma räägin sulle, kuidas asju tegema peab. Aga milles?
	\item Hansa pangatehnoloogia näide: organisatsioonis peab olema äri põhjalikult tundev inimene. See ei pruugi olle iseenesest mõistetav
	\item Korrelatsioon: kui inimene ärist aru saab, on ta piisavalt nutikas ka ITst aru saama. Ükskõik millest hästi aru saamine eeldab reeglina õppimisvõimet ja vajalikud IT-teadmised ei ole väga keerulised. On omandatavad paari ämbrisseastumise järel.
\end{itemize}

\subsection{Millal on tarkvara valmis?}
Tarkvara on valmis siis, kui tehakse vastav otsus. Arendusprotsess vähegi mittetriviaalsel juhul ei koondu. Järelikult
\begin{itemize}
	\item Peab olema võimekus otsust teha
	\item Peab eksisteerima otsustusprotsess
	\item Peab otsustajatel olema mandaat ja pädevus otsustada. Seega peab otsustusprotsess kaasama kõiki osapooli (klient, opsid, ka arendaja) ja olema läbipaistev
\end{itemize}

Saalist
\begin{itemize}
	\item Kui tellija nõuded on täidetud. Aga mis on nõuete kvaliteet? Nii täpselt saabki siis hinnata seda kvaliteeti
	\item Kui tarkvara on maha kantud! Viide arenduse ja halduse kuludele eespool
\end{itemize}

\subsection{Milline tarkvara on kvaliteetne?}
\TODO täida sisuga
Saalist
\begin{itemize}
	\item Funktsionaalsus on olemas
	\item Tellija ütleb nii
	\item Tellija, arendaja, ops ja infosec (miks mitte ka täitja) ütleb nii. Kvaliteet kui kokkulepe!
\end{itemize}

\subsection{Mis on teenus?}
ITIL\index{ITIL} defineerib teenuse mõiste omandi läbi: [teenus on] \enquote{väärtuse pakkumine klientidele võimaldades neil saavutada lõpptulemusi ilma kaasnevaid kulusid ja riske kandmata.}\cite{itil}. Ehk, teenust pakkudes võimaldadatakse kellelgi saada väärtust ilma, et ta peaks teatud kulusid ja riske ise kandma. Mõiste on olemuselt tautoloogiline sisaldades nii abstraktse kui kui konkreetse väärtuse pakkumist: teenuseid pakkudes pakutakse tegelikult kulude ja riskide kandmise teenust. Järelikult on teenuse pakkumise eelduseks võimekus mingil põhjusel pakkuda väiksemaid riske ja kulusid. Kuidas seda saavutatakse ning mis \enquote{teenus} tegelikult on, ei öelda. Siiski on definitsioonis olulisel kohal väärtuse ja seega ka kliendi mõiste, teenust ei saa osutada ilma kliendita.

ISO 20000\index{ISO 20000} on märksa ebamäärasem defineerides teenuse kui vahendid või meetodid, mida organisatsioonid kasutavad klientide poolt väärtustatud ja soovitud tulemuste saavutamiseks. \cite{iso20000}. Kuigi mängus on klient, ei ole enam tehtud eeldust, et klient teenuse tarbimise läbi rohkem väärtust saab kui tulemust ise tekitades. Oluline on ka tähele panna, et kui ITIL keskendub funktsioonile (\enquote{väärtuse pakkumine läbi...}) siis ISO vormile (\enquote{vahendid või meetodid}. 

Kindlasti on mitmesuguseid definitsioone veel, kuid ilmsesti on need kaks kõige levinumad ja osundavad kenasti levinud probleemile. Nimelt on tegemist juhtimise seisukohalt suhteliselt kasutute määratlustega. Neile on võimalik üles ehitada küll keerukas ja kõikehõlmav teoreetiline aparatuur kuid nad on praktiliseks kasutamiseks liiga hägusad. Kui teenus on \enquote{väärtuse pakkumine väärtuse pakkumiseks vajalike vahendite abil}, siis on see kahtlemata õige aga kuidas ja mis otsuseid me selle määratluse abil tegema peaksime? Kogu IT pakutav on vaadeldav kui üks teenus ja kindlasti on ka kuskil serveris käiv \texttt{cron} \enquote{vahend väärtuse pakkumiseks}. 

Mida siis teha? IT juht tegeleb reeglina mitte üksikute teenuste vaid teenuste portfelliga. Portfelli kui terviku ja selle üksikute elementide jõudlust peaks olema võimalik hinnata ning teha otsuseid. On selge, et ühest elemendist koosnev teenusteportfell lükkab teenustega tegelemise lihtsalt kellegi teise õlule. Samuti on ilmselt igal inimesel ja organisatsioonil konkreetses tehnoloogilises kontekstis võimalik hoomata piiratud hulka teenuseid. Rohkemaid teenuseid küll võib defineerida aga sellest ei saa palju kasu olla: juhtimislikke otsuseid on võimalik teha ainult hoomatavuse piires. 

Seega jõuame veidi praktilisema teenuse definitsioonini: \enquote{Teenus on miski, millest ja millesarnastest on võimalik koostada praktiliselt kasulik teenuste portfell}. Seejuures on täitsa ükskõik, kas lähtutakse ITILi, ISO või mõnest kolmandast teenuse definitsioonist. Samuti on oluline mõista, et eri organisatsioonide portfellijuhtimise võimekus on erinev. Hästiarenenud EA\index{Enterprise Architecture} taristu abil on võimalik tegeleda tuhandete teenustega samas kui käsitsi toimetades võib ka kümmekond teenust üle jõu käia.

\TODO Saalist: Väärtuse edastamise protsess. Aga väärtus on subjektiivne. Ja seega on teenused subjektiivsed. Milline paljudest valida? Tuleb ise otsustada, väärtuse järgi. Ja otsust üle vaadata, kui vajadustele ei vasta.

\TODO inkorporeeri \cite{svcthesis} peatükk 1.2 