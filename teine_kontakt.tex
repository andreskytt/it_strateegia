
\begin{document}

\maketitle

\section{Sissejuhatus}
\begin{frame}[fragile]
  \frametitle{Eelmine kord}
  Saame tuttavaks ja räägime strateegiast üldiselt
	\begin{itemize}
		\item Sissejuhatus: kes ma olen ja kuidas minu käest hinde saab
		\item IT seos äristrateegiaga, IT mõju ärile
		\item Süsteemidest ja tarkvara arhitektuurist
	\end{itemize}
\end{frame}

\begin{frame}[fragile]
  \frametitle{Täna kavas}
		\includegraphics[width=\textwidth]{aine_struktuur_teine.pdf}
\end{frame}

\section{Äriplaani koostamine}
\begin{frame}[fragile]
  \frametitle{Äriplaan}
	\begin{itemize}
		\item IT kulumudel. Arendus ei ole mäluga, haldus on
		\item FFC mudeli ja ärimudeli seos
		\item Tehnilise võla olemus
	\end{itemize}
\end{frame}


%Arutelu koht
\begin{frame}[fragile]
  \frametitle{Arutelu koht}
		\begin{center}
			\textbf{Kui palju peaks juhtkond teadma tehnilisest võlast?}
		\end{center}
\end{frame}

\section{Tarkvaratehnika}
\begin{frame}[fragile]
  \frametitle{Tarkvaratehnika}
	\begin{itemize}
		\item Software engineeringu mõiste
		\item Miks on tarkvaratehnika oluline? \cite{spolsky2004joel2}
		\item Brooks ja tema "No silver bullet" \cite{brooks1975mythical}
	\end{itemize}
\end{frame}


%Arutelu koht
\begin{frame}[fragile]
  \frametitle{Arutelu koht}
		\begin{center}
			\textbf{Mis teeb programmeerimise keeruliseks?}
		\end{center}
\end{frame}

\section{Tarkvaratehnika}
\begin{frame}[fragile]
  \frametitle{Tarkvaratehnika}
	\begin{itemize}
		\item Mõistliku koodi 12 reeglit \cite{spolsky2004joel}
		\item Tehnilise võla olemus ja struktuur \cite{fowlerdebt}. Tehniline võlg tuleb ärist (näita kuidas kvadranditi) ja lõpeb äriga aga vahepeal läbib IT.
	\end{itemize}
\end{frame}

%Arutelu koht
\begin{frame}[fragile]
  \frametitle{Arutelu koht}
		\begin{center}
			\textbf{Kui palju peaks juhtkond teadma tehnilisest võlast?}
		\end{center}
\end{frame}

\section{IT valitsemine}
\begin{frame}[fragile]
  \frametitle{IT valitsemine}
	\begin{itemize}
		\item Management vs. leadership. Manager is expendable, leader is not
		\item Mida me juhime, kui me juhime ITd? Inimesed, ressursid, protsessid
		\item Policy resistance olemus?
	\end{itemize}
\end{frame}

\begin{frame}[fragile]
  \frametitle{Inimesed}
	\begin{itemize}
		\item Inimesed käituvad viisil, mis maksimeerib nende huvisid. Kas need huvid on ettevõtte huvidega sobivad, on juhi mure
		\item Enne, kui inimesi valitsema/kontrollima hakata, küsi: kes läheb ära ja kes saab edukaks. \emph{Mitte} "kes jääb" 
	\end{itemize}
\end{frame}


%Arutelu koht
\begin{frame}[fragile]
  \frametitle{Arutelu koht}
		\begin{center}
			\textbf{Kuidas mõõta programmeerija tulemust?}
		\end{center}
\end{frame}

\begin{frame}[fragile]
  \frametitle{Ressursid}
	\begin{itemize}
		\item Kriitiline vahe riigi- ja erasektori vahel ressursijuhtimisel
		\item Ülesanne on alati maksimeerida bang per buck. Kõik muu on irrelevantne
		\item Seos tehnilise võlaga. Veendu, et tellija saab aru oma otsuste pikaajalisest rahalisest mõjust
		\item Strateegilises plaanis ei ole ressursijuhtimise protsessid kriitilised, osta HR paperwork ja finants väljast
	\end{itemize}
\end{frame}

\begin{frame}[fragile]
  \frametitle{Protsessid}
	\begin{itemize}
		\item Küberneetika kontrolli probleem. 
		\item Süsteemide loomulik inerts, eri tüüpi tasakaalud, beer game
		\item Jaapani mõtteviis (high velocity edge viited)
	\end{itemize}
\end{frame}


%Arutelu koht
\begin{frame}[fragile]
  \frametitle{Arutelu koht}
		\begin{center}
			\textbf{Kui palju peaks juhtkond teadma tehnilisest võlast?}
		\end{center}
\end{frame}

\begin{frame}[fragile]
  \frametitle{Arenduse pipeline}
	\begin{itemize}
		\item Miks see on oluline (kuigi läheb natuke taktikaliseks)
		\item Printsiibid: eelarve- ja otsusepõhine juhtimine. Nende tasakaalust. Suurte ja väikeste kivide probleem (seos tehnilise võlaga)
		\item Capacity, overflow ja pipeline otsused. Demand ületab alati supply. Järelikult jääb asju alati pipeline lõppu. On kriitiline, kuidas neid sealt välja võetakse
		\item Kuidas teha. Mõlemalt poolt mandaadiga juhid, regulaarne (veel parem: online), rolling, mitte spot
	\end{itemize}
\end{frame}

%Arutelu koht
\begin{frame}[fragile]
  \frametitle{Arutelu koht}
		\begin{center}
			\textbf{Mida teha, kui klient ei kuula?}
		\end{center}
\end{frame}


\section{Viited}

\begin{frame}[t,allowframebreaks,]
  	\bibliographystyle{plainnat}
	\bibliography{it_strateegia} 

\end{frame}

%\plain{Küsimusi?}
\begin{frame}[plain]
	\begin{center}Küsimusi?\end{center}
\end{frame}

\end{document}