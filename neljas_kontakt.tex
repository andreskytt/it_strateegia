\begin{document}

\maketitle

\section{Sissejuhatus}
\begin{frame}[fragile]
  \frametitle{Eelmine kord}
  Keerulised teemad, rõhk keeruliste süsteemide käitumisel eri olukordades
	\begin{itemize}
		\item Jätkusuutlikkus on oluline teema, kui pealispinna alla vaadata
		\item Muudatused äris on vältimatud, nende tagajärjed ITle mittetriviaalsed
		\item Riskijuhtimine klassikalisel kujul keeruliste süsteemide puhul ei toimi
	\end{itemize}
\end{frame}

\begin{frame}[fragile]
  \frametitle{Täna kavas}
		\includegraphics[width=\textwidth]{aine_struktuur_neljas.pdf}
\end{frame}
\section{Infoturve}
\begin{frame}[fragile]
  \frametitle{Infoturve}
	\begin{itemize}
		\item Riskijuhtimise kordus: ei ole intsidente, on pidev voog ründeid
		\item Kuna tarkvarakomponendid ei ole turvalised ja nende turvalisusest ka ei piisa, tuleb truvalisusele läheneda süsteemselt
		\item Arhitektuuri roll infoturbes: safe by design. Korea isikukoodi näide
		\item Must turg: küsimus on äris, kehtivad turu reeglid
		\item Pen test: küsimus ei ole sisse saamises, küsimus on kiiruses, hinnas ja sügavuses. Kui otse rünnatakse, siis ka maha võetakse
		\item Strateegiline roll on kogukonnal: osa süsteemi elemente ei ole teie kontrolli all, tundliku info jagamise aluseks on usaldus, õige käitumise aluseks on õigeaegne info, üksi ei saa asju ajada (Seepärast CERTide võrk tekkiski). Eestis küberkaitseliit. Laiemalt konverentsid (ja seal esinemine), inimeste harimine (et oleks, kes kogukonnas osaleks) 
	\end{itemize}
\end{frame}


%Arutelu koht
\begin{frame}[fragile]
  \frametitle{Arutelu koht}
		\begin{center}
			\textbf{1. küsimus}
		\end{center}
\end{frame}

\section{Partnerite juhtimine}
\begin{frame}[fragile]
  \frametitle{Partnerite juhtimine}
	\begin{itemize}
		\item Hinda realistlikult oma võimet partnerit juhtida. Jaapani näide. 
		\item Kui vähegi kahtlus, ära juhi partnerit. Sun Tzu tsitaat. Strateegiad:
			\begin{itemize}
				\item Ignoreeri kogu probleemi (kohvi ja vetsupaber)
				\item Juhi tehnoloogiat
				\item Juhi/kasuta standardeid
				\item Juhi kogukonda
				\item Jaga ja valitse
			\end{itemize}
	\end{itemize}
\end{frame}

%Arutelu koht
\begin{frame}[fragile]
  \frametitle{Arutelu koht}
		\begin{center}
			\textbf{2. küsimus}
		\end{center}
\end{frame}

\begin{frame}[fragile]
  \frametitle{Partnerite juhtimine}
  Kui muud moodi ei saa
	\begin{itemize}
		\item Räägi riskijuhtidega. Partneril on strateegiline roll riskides
		\item Ökosüsteemi, mitte üksiku partneri vaade. Kes veel on mängijad, mis on nende huvid?
		\item Partneri juhtimise aluseks on võime ennast juhtida. Nagu ka inimeste puhul. Rõhuv enamus "partner on loll" olukordi minu kogemuses tulenevad otseselt kliendi võimekusest kliendina käituda. Kord kaotatud strateegilist positsiooni on väga raske tagasi võita (teadmuskao tagasiside)
		\item Kliendina käitumiseks kriitiline
			\begin{itemize}
				\item Kontroll IP üle. Nii tehniline kui sisuline!
				\item Kontroll arhitektuuri üle
				\item Kontroll lepingu ja selle täitmise üle. Iga ebatäpsust siin kasutatakse ära
			\end{itemize}		
		\item Kehtestav käitumine: oma vajaduste selge ja järjekindel väljendamine ning nende eest seismin
	\end{itemize}
\end{frame}

%Arutelu koht
\begin{frame}[fragile]
  \frametitle{Arutelu koht}
		\begin{center}
			\textbf{3. küsimus}
		\end{center}
\end{frame}

\section{Kvaliteedi juhtimine}

\begin{frame}[fragile]
  \frametitle{Kvaliteedi juhtimine}
	Miks tarkvaras on vead
	\begin{itemize}
		\item Vigade parandamise dünaamiline mudel, selle matemaatiline lahend
		\item Mudeli järelmid: alati on veel vigu, testima tuleb hakata vara
		\item Otsustuskriteeriumid lõpetamiseks: eraldame ratsionaalse ja irratsionaalse. Ratsionnalselt (graafik): kui oleme jõudnud allapoole spetsifikatsiooni veamarginaali
	\end{itemize}
\end{frame}

%Arutelu koht
\begin{frame}[fragile]
  \frametitle{Arutelu koht}
		\begin{center}
			\textbf{4. küsimus}
		\end{center}
\end{frame}

\begin{frame}[fragile]
  \frametitle{Kvaliteedi juhtimine}
	Jaapanlaste lähenemine
	\begin{itemize}
		\item Seos lean liikumisega
		\item Kaizeni olemus
		\item Teede parandamise näide
		\item Selle järeldused tarkvaratehnika ja QA jaoks
	\end{itemize}
\end{frame}

%Arutelu koht
\begin{frame}[fragile]
  \frametitle{Arutelu koht}
		\begin{center}
			\textbf{5. küsimus}
		\end{center}
\end{frame}


\section{Teenustaseme juhtimine}
\begin{frame}[fragile]
  \frametitle{Teenustaseme juhtimine}
	\begin{itemize}
		\item Teenustaseme juhtimise aluseks on teenuse mõiste olemasolu (alignment kõigis kihtides)
		\item Arutelu teenustasemete üle riskijuhtimist kaasamata on lihtsalt grupp jonnivaid inimesi
		\item Teenustase tuleb määratleda läbi riskide
		\item Järelikult tuleb mõõta teenust, mitte masinaid. Masinad, nagu rääkisime, lähevad ennustataval viisil katki
		\item Keerulist süsteemi modelleerima hakka ainult siis, kui sul on palju aega ja raha ning tõsine vajadus. Sest ta on keeruline. Ja, nagu rääkisime, teenustaseme kao põhjus võib olla mõõtmisvea lähedal ja seega raskesti mürast eristatav
	\end{itemize}
\end{frame}


%Arutelu koht
\begin{frame}[fragile]
  \frametitle{Arutelu koht}
		\begin{center}
			\textbf{6. küsimus}
		\end{center}
\end{frame}


\section{Viited}

\begin{frame}[t,allowframebreaks,]
  	\bibliographystyle{plainnat}
	\bibliography{it_strateegia} 

\end{frame}

%\plain{Küsimusi?}
\begin{frame}[plain]
	\begin{center}Küsimusi?\end{center}
\end{frame}


\end{document}