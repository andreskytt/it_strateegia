\documentclass{article} 
\usepackage[estonian]{babel}
%\usepackage{fontspec} 
\usepackage{graphicx}
\usepackage{hyperref}
\usepackage{natbib}
\usepackage{booktabs}
\usepackage{makeidx}
\usepackage{tgpagella}
\usepackage[T1]{fontenc}

\hypersetup{
    colorlinks=true,       % false: boxed links; true: colored links
    linkcolor=black,          % color of internal links
    citecolor=black,        % color of links to bibliography
    filecolor=magenta,      % color of file links
    urlcolor=black           % color of external links
}

\renewcommand\refname{Viited}

\makeatletter 
\def\s@btitle{\relax} 
\def\subtitle#1{\gdef\s@btitle{#1}} 
\def\@maketitle{% 
  \newpage 
  \null 
  \vskip 2em% 
  \begin{center}% 
  \let \footnote \thanks 
    {\LARGE \@title \par}% 
                \if\s@btitle\relax 
                \else\typeout{[subtitle]}% 
                        \vskip .5pc 
                        \begin{large}% 
                                \textsl{\s@btitle}% 
                                \par 
                        \end{large}% 
                \fi 
    \vskip 1.5em% 
    {\large 
      \lineskip .5em% 
      \begin{tabular}[t]{c}% 
        \@author 
      \end{tabular}\par}% 
    \vskip 1em% 
    {\large \@date}% \\
    \vfill
  \end{center}% 
  \par 
  \vskip 1.5em} 
\makeatother 


\title{IFI7013. IT Strateegiline Juhtimine}
\subtitle{Slaidid ja märkmed}
\date{\today}
\author{Andres Kütt}
%\institute{Riigi Infosüsteemi Amet}

\newcounter{slidenum}
\setcounter{slidenum}{2} % set to 2 if want to exclude title page of presentation

\newcommand\showslide{
%  \clearpage 
  \begin{center}
    \framebox{\includegraphics[page=\arabic{slidenum},width=.99\textwidth]{esimene_kontakt_beamer.pdf}}
%	\includegraphics[page=\arabic{slidenum},width=.65\textwidth]{esimene_kontakt_beamer.pdf}
    \stepcounter{slidenum}
  \end{center}
  \clearpage
}

\begin{document}
\maketitle
\clearpage
\section{Tarkvaratehnika}
\subsection{Kood ei roosteta. Või siiski?}
Joel Spolsky ütleb, et reeglina on väga halb mõte oma koodibaasi ümber kirjutada, sest kood ju "ei roosteta" \citep{joelrust}. Ta toob mitu näidet väga ebameeldivate tagajärgedega ümber-kirjutamis ettevõtmiste kohta ning, tõesti, neid on ka siinkirjutaja praktikas mitmeid ette tulnud. Põhjuseid on mitmeid, peamiseks ehk paratamatu teadmuskadu: iga tükki koodi on juba enne rakenduse valmimist kümneid kui mitte sadu kordi muudetud ja parandatud parandamaks vigu, ületamaks nõuete ebatäielikkust jne. Nii kaob igasugune võimalus hinnata, kas kood on selline, nagu ta on, põhjusega või põhjuseta. Rääkimata analüüsist, kas põhjus jätkuvalt kehtib. Miks siis tekib vahel siiski kohu asju ümber teha ning miks on Eestis kehtestatud \emph{no legacy policy}?

Ühest küljest on asja taga kindlasti programmeerijad. Nagu Spolsky õigesti osundab, on koodi lugemine palju keerulisem, kui selle kirjutamine. Seega, eriti kui tegu on kellegi teise koodiga, on programmeerijale oluliselt lihtsam kirjutada uus kood kui üritada vanast aru saada. Loomulikult tõlgitakse vahe rahanumbriks ning uue süsteemi ehitamine võib vana turgutamisest oluliselt odavam näida. Erinevalt uue süsteemi ehitamisest, ei ole vana muutmine kergesti hinnatav ning tellija ees on kas väike fikseeritud number ebamäärase kahjuga või suur riskantne number ebamäärase tuluga. 

Teisalt võib ümberkirjutamissoovi taga olla lihtne äriliste riskide vähendamine. Vana kood peidab endas alati üllatusi ning riskide vähendamiseks võib pakkuja eelistada uue kirjutamist. 

Mõlemal juhul tekib lihtsasti olukord, kus tellijal ja otsuse tegijal ei ole piisavalt tehnilist teadmist ja/või informatsiooni vana süsteemi kohta. Erinevalt koodist teadmus kindlasti kõduneb. Sel puhul on mõistlik käivitada väikesemahuline konkreetsete tulemustega piloot rakenduse kvaliteedi hindamiseks. Selle lõppedes on nii tellijal kui täitjal palju selgem ülevaade, kui keeruline vana koodibaasi putitamine tegelikult on.

Olemasoleva koodi puhul võib tegu olla ka "pusaga": süsteemiga, mis on aja jooksul kas arhitektuuriliselt või tehnoloogiliselt keeruliseks kasvanud. Nii arhitektuuri kui tehnoloogia puhul on kindlasti tegu ka mööduva moega, tehno\-loogiad ning arhitektuurimustrid vananevad. Samas ei ole kindlasti tegu \textit{carte blanche} põhjusega rakendusi ümber kirjutada, COBOLi süsteeme on edukalt veebirakendustega integreeritud. Jällegi on mõistlik läbi viia piloot ning teha otsus kindla teadmise, mitte kellegi arvamuse pinnalt.

\begin{figure}[htp]
	\begin{center}
		\includegraphics[height=4cm]{spaniel.jpg}\includegraphics[height=4cm]{wolf.jpg}\includegraphics[height=4cm]{mastiff.jpg}
		\caption{Spanjel, hunt ja mastif}
	\end{center}
\end{figure}

Lisaks negatiivsele on koodi uuesti kirjutamisel ka üks oluline positiivne omadus. Ta annab võimaluse innovatsiooniks ning asjade puhtalt lehelt uuesti mõtestamiseks. Jah, spanjelist on ilmselt võimalik mastifi-laadne elukas aretada aga võibolla on efektiivsem alustada siiski nende ühisest eellasest, hundist? Kindlasti tuleks keskenduda mitte tehnilisele vaid ärilisele innovatsioonile ümber mõtestades äriprotsesse, automatiseerides ning efektiivistades. Seejuures on muidugi eelduseks, et meil on piisavalt aega ja raha seda mõttetööd põhjalikult ette võtta ning et on alust eeldada, et tulemus praegusest olukorrast oluliselt erineb.

Lõpuks tuleb panna kõrvuti süsteemi ümber kirjutamise kulu (korrutades esiaglse hinnangu vähemalt kahega), olemasoleva muutmise ning mõlema alternatiivi halduskulude nüüdisväärtus. Kui nüüd tundub, et rakenduse uuesti kirjutamine on siiski mõistlik, on oluline aru saada, miks nii läks. Jällegi Joelile toetudes, ei ole mõistlik eeldada, et kui ühel korral ei õnnestunud hankida mõistlikku süsteemi või seda pusaks muutumast hoida, siis teisel korral asjad teisiti lähevad. On oluline, et suudetakse välja tuua konkreetsed tegevused, mille abil hoidutakse vajadusest süsteem uuesti ümber kirjutada. Siinkohal kuuleb ilmselt argumenti \emph{build one to throw away} aga sel juhul peaks olema võimalik vähemalt üles kirjutada, mida esimesest korrast täpselt õpiti.

\section{IT valitsemine}
\subsection{Inimeste valitsemine}
Valitsemise üsna lahutamatu osa on paraku vajadus inimesed organisatsioonist eemaldada. Jättes kõrvale juriidilised ja muud nüansid, on oluline aru saada, et mõtteliidri (\emph{thought leader}) lahkumine on meeskonna dünaamika jaoks oluline sündmus. Samuti on oluline mõista, et tegu on vältimatu protsessiga, mis seega on mõistlik läbi viia kontrollitult ja minimaalsete kadudega. Järgnevas selleks mõned näpunäited: 

\begin{itemize}
	\item Vii inimene organisatsioonist välja enne, kui tema vastuolud muutuvad meeskonna vastuoluks. Definistiooni järgi inimesed järgnevad liidrile ning kui tollel on kellegagi vastuolud, kanduvad nood paratamatult varem või hiljem meeskonda. Kui hiljaks jääda lahkuvad (või tuleb vastuseisu tõttu eemaldada) ka teised tiimi liikmed peale juhi. Nii võib oskamatu käitumisega vallandada lumepalliefekti, mis organisatsiooni kiiresti ajudest tühjendab.
	\item Väike (!) hulk võtmeisikuid peab plaanist ette teadma, sealhulgas muidugi ka eemaldatav ise. Nii on neil ühest küljest võimalik tulevaseks kriisiks valmistuda kuid teisalt saab nii vältida emotsionaalset avalikku käitumist ning muidu üllatusi. Etteteatamisaeg võib ulatuda mõnest päevast mõne tunnini. Pikem aeg tekitab permanentse kriisi olukorra ja kommunikatsioon väljub kontrolli alt
	\item Kommunikatsiooni kolm sammu. Kõik sammud läbitakse väga lühikeste (kõige rohkem mõned tunnid) intervallidega vältimaks uudiste lekkimist (ingl. \emph{techcrunch meltdown} tuntud tehnoloogiauudiste portalli järgi) ning minimeerimaks kriisi kestvust. 
		\begin{enumerate}
			\item "Vanaisa"\footnote{Juhi juht. Selles kontekstis isik, kes teeb otsuse inimene välja viia.} teade. Nii antakse teada, kes olukorda kontrollib. Sisaldab
				\begin{itemize}
					\item Kes lahkub millal ja miks. Tekst olgu viisakas, inimesed kas teavad niigi või suudavad ridade vahelt lugeda
					\item Mis juhtub järgmisena ja millal. Inimestele ei meeldi ebakindlus, hirmud tuleb maha võtta. Oluline punkt siin: kes ja millal asendab lahkuja?
					\item Muutused organisatsiooni toimimises. Lahkuja oli kindlasti osaline mingites äriprotsessides (koodi ülevaatused, tarnete vastu võtmine jms.), kuidas need edasi toimima hakkavad?
				\end{itemize}
			\item Lahkuja teade. Tavaliselt suunatud lähematele kolleegidele. Võib olla suhteliselt otsekohene aga kui on karta midagi mürgist, võib (viisakalt) paluda teksti kooskõlastamist
			\item Avalik teade. Kui vähegi usutav, võiks tekst olla kiitvas, positiivses ja tänulikus toonis. Kui mitte, siis lakooniline kuiv tekst.
		\end{enumerate}
\end{itemize}


%\showslide
%\showslide
%\lipsum[1-5]

%\showslide
%\lipsum[2]
\nocite{*}
\bibliographystyle{plainnat}
\bibliography{it_strateegia} 

\end{document}
