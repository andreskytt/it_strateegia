\documentclass{article} 
\usepackage{graphicx}
\usepackage{hyperref}
\usepackage{natbib}
\usepackage{booktabs}
\usepackage{makeidx}
\usepackage{tgpagella}
\usepackage[T1]{fontenc}

\hypersetup{
    colorlinks=true,       % false: boxed links; true: colored links
    linkcolor=black,          % color of internal links
    citecolor=black,        % color of links to bibliography
    filecolor=magenta,      % color of file links
    urlcolor=black           % color of external links
}

\renewcommand\refname{Viited}

\makeatletter 
\def\s@btitle{\relax} 
\def\subtitle#1{\gdef\s@btitle{#1}} 
\def\@maketitle{% 
  \newpage 
  \null 
  \vskip 2em% 
  \begin{center}% 
  \let \footnote \thanks 
    {\LARGE \@title \par}% 
                \if\s@btitle\relax 
                \else\typeout{[subtitle]}% 
                        \vskip .5pc 
                        \begin{large}% 
                                \textsl{\s@btitle}% 
                                \par 
                        \end{large}% 
                \fi 
    \vskip 1.5em% 
    {\large 
      \lineskip .5em% 
      \begin{tabular}[t]{c}% 
        \@author 
      \end{tabular}\par}% 
    \vskip 1em% 
    {\large \@date}% \\
    \vfill
  \end{center}% 
  \par 
  \vskip 1.5em} 
\makeatother 


\title{IFI7013. IT Strateegiline Juhtimine}
\subtitle{Slaidid ja märkmed}
\date{\today}
\author{Andres Kütt}
%\institute{Riigi Infosüsteemi Amet}

\newcounter{slidenum}
\setcounter{slidenum}{2} % set to 2 if want to exclude title page of presentation

\newcommand\showslide{
%  \clearpage 
  \begin{center}
    \framebox{\includegraphics[page=\arabic{slidenum},width=.99\textwidth]{esimene_kontakt_beamer.pdf}}
%	\includegraphics[page=\arabic{slidenum},width=.65\textwidth]{esimene_kontakt_beamer.pdf}
    \stepcounter{slidenum}
  \end{center}
  \clearpage
}

\begin{document}
\maketitle
\clearpage
\section{IT valitsemine}
\subsection{Inimeste valitsemine}
Valitsemise üsna lahutamatu osa on paraku vajadus inimesed organisatsioonist eemaldada. Jättes kõrvale juriidilised ja muud nüansid, on oluline aru saada, et mõtteliidri (\emph{thought leader}) lahkumine on meeskonna dünaamika jaoks oluline sündmus. Samuti on oluline mõista, et tegu on vältimatu protsessiga, mis seega on mõistlik läbi viia kontrollitult ja minimaalsete kadudega. Järgnevas selleks mõned näpunäited: 

\begin{itemize}
	\item Vii inimene organisatsioonist välja enne, kui tema vastuolud muutuvad meeskonna vastuoluks. Definistiooni järgi inimesed järgnevad liidrile ning kui tollel on kellegagi vastuolud, kanduvad nood paratamatult varem või hiljem meeskonda. Kui hiljaks jääda lahkuvad (või tuleb vastuseisu tõttu eemaldada) ka teised tiimi liikmed peale juhi. Nii võib oskamatu käitumisega vallandada lumepalliefekti, mis organisatsiooni kiiresti ajudest tühjendab.
	\item Väike (!) hulk võtmeisikuid peab plaanist ette teadma, sealhulgas muidugi ka eemaldatav ise. Nii on neil ühest küljest võimalik tulevaseks kriisiks valmistuda kuid teisalt saab nii vältida emotsionaalset avalikku käitumist ning muidu üllatusi. Etteteatamisaeg võib ulatuda mõnest päevast mõne tunnini. Pikem aeg tekitab permanentse kriisi olukorra ja kommunikatsioon väljub kontrolli alt
	\item Kommunikatsiooni kolm sammu. Kõik sammud läbitakse väga lühikeste (kõige rohkem mõned tunnid) intervallidega vältimaks uudiste lekkimist (ingl. \emph{techcrunch meltdown} tuntud tehnoloogiauudiste portalli järgi) ning minimeerimaks kriisi kestvust. 
		\begin{enumerate}
			\item "Vanaisa"\footnote{Juhi juht. Selles kontekstis isik, kes teeb otsuse inimene välja viia.} teade. Nii antakse teada, kes olukorda kontrollib. Sisaldab
				\begin{itemize}
					\item Kes lahkub millal ja miks. Tekst olgu viisakas, inimesed kas teavad niigi või suudavad ridade vahelt lugeda
					\item Mis juhtub järgmisena ja millal. Inimestele ei meeldi ebakindlus, hirmud tuleb maha võtta. Oluline punkt siin: kes ja millal asendab lahkuja?
					\item Muutused organisatsiooni toimimises. Lahkuja oli kindlasti osaline mingites äriprotsessides (koodi ülevaatused, tarnete vastu võtmine jms.), kuidas need edasi toimima hakkavad?
				\end{itemize}
			\item Lahkuja teade. Tavaliselt suunatud lähematele kolleegidele. Võib olla suhteliselt otsekohene aga kui on karta midagi mürgist, võib (viisakalt) paluda teksti kooskõlastamist
			\item Avalik teade. Kui vähegi usutav, võiks tekst olla kiitvas, positiivses ja tänulikus toonis. Kui mitte, siis lakooniline kuiv tekst.
		\end{enumerate}
\end{itemize}


%\showslide
%\showslide
%\lipsum[1-5]

%\showslide
%\lipsum[2]
\nocite{*}
\bibliographystyle{plainnat}
\bibliography{it_strateegia} 

\end{document}
